\section{Architectural Design and Technical Considerations}
\label{sec:architectural_design}


\subsection{Clang-based Architectural Design for Hylo's C Interoperability Tooling}

To address the challenges outlined above, we propose a Clang-based architectural design for Hylo's C interoperability tooling. This architecture prioritizes fidelity and automation by leveraging the rich semantic information provided by an industrial-strength compiler front-end.

\subsubsection{Parsing Technology: A Clang-Based Approach}
Generating accurate bindings requires a high-fidelity parser for the C language. While various tools exist, a comparative analysis presented in \autoref{app:c_ast_introspection_tools} reveals that leveraging the Clang compiler front-end is the most robust approach.
\subsubsection{Architecture of Customizabile Mappings}
TODO compare comment-based vs attribute based mappings, and describe rule files

TODO move this to the relevant section

% \subsubsection{FFI Generation vs Direct AST Emission}
% TODO add nice diagram of the two types of architectures


\subsubsection{Declarations vs Opaque data + computed properties}
This is important, as unions, bit-fields and other C layout constructs that don't have Hylo representations can be represented as opaque data with computed properties accessing them just like other regular Hylo types. (todo describe problems of separate declarations vs computed properties)

TODO shouldn't be here because it's discussed in the next section in detail.

\subsubsection{Lowering Function Calls and Handling the ABI}

Reimplementing the complex ABI logic for every supported platform is a monumental and error-prone task. Instead, our proposed architecture leverages Clang's own ABI handling capabilities. Since Hylo uses LLVM as its backend, and Clang is also an LLVM-based compiler, we can delegate the responsibility of ABI-correct function lowering to Clang. The interoperability tool can instruct Clang to generate the LLVM Intermediate Representation (IR) for a call to a C function, ensuring that all platform-specific calling conventions and data layout rules are correctly applied. This approach significantly reduces implementation complexity and ensures correctness by relying on the same battle-tested ABI logic used by Clang itself.

TODO discuss what responsibilities are there to correctly call functions and lay out structures according to the platform ABI

todo: describe what are the inherent things that need to be different in ABIs based on different architectures, and what are imposed by the OS and the C language. Mention the problem of compilers incorrectly implementing ABIs which sometimes necessitates the interop tool to implement workarounds for specific compilers or platforms (link Emilio interview).

todo: describe the option to add wrapper function


\subsubsection{IDE Integration}

A seamless developer experience requires that IDE features like code completion, go-to-definition, and renaming work across the C/Hylo boundary. Our design proposes the creation of a Hylo Language Server that acts as a proxy, coordinating with \texttt{clangd}, the official C/C++ language server for the LLVM project. When a request is made for a symbol defined in C, the Hylo LSP would forward the request to \texttt{clangd} and translate the response. This architecture, inspired by Swift's SourceKit-LSP, allows Hylo to benefit from the full power of \texttt{clangd} without having to reimplement its functionality.

TODO include nice diagram, illustrate how symbol renaming works as an example, and check if modifications are needed in \texttt{clangd} or it can be used as is.


\subsection{Customization and Ambiguous Mappings}

A core principle of our design is acknowledging that a single, direct mapping is often insufficient. Many C constructs are used to represent different semantic ideas. A prime example is the C \texttt{enum}. It can represent:
\begin{enumerate}
    \item A set of mutually exclusive cases \(\rightarrow\) translate to a Hylo enum for proper exhaustiveness checking
    \item A collection of independent bitflags \(\rightarrow\) translate to an \texttt{OptionSet} \cite{OptionSet} for easy combination and checking of binary flags.
    \item A simple group of named integer constants. \(\rightarrow\) translate to a Hylo namespace with constants.
\end{enumerate}
A rigid mapping to a single Hylo construct would be un-idiomatic in at least two of these cases. Therefore, our design proposes a system of \textbf{sensible defaults with user-customizable overrides.} By default, an \texttt{enum} might be imported as a struct of static constants, which is always safe. However, the developer can provide an annotation (e.g., in a separate configuration file or directly in the C header if they have ownership) to guide the mapping:

This principle of user-guided mapping extends to other areas, such as specifying that a C function returning an integer error code should be imported as a Hylo function that \texttt{throws}. This empowers developers to create bindings that are not only correct but also safe and highly idiomatic.

TODO describe the rules file vs inline annotations (comments vs attributes)