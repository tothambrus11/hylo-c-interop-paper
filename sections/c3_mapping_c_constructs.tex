\section{Mapping C Constructs to Hylo}
\label{sec:mapping_c_constructs}

This section defines the core of our contribution: the rules for translating C declarations into idiomatic Hylo constructs.

\subsection{Primitive Types}
\subsubsection{Integer Types}
\label{ssec:integer_types_mapping}
The mapping of primitive types is foundational to interoperability. Mapping fixed-size integers (\texttt{uint8\_t}, \texttt{int32\_t}, etc.) and word-sized integers (\texttt{size\_t}/\texttt{ssize\_t}) is trivial, as they have an exact corresponding type in Hylo. A key challenge is that the size of C's standard integer types (\texttt{char}, \texttt{short}, \texttt{int}, \texttt{long}) are platform-dependent. For example, \texttt{char} is defined as an at least 8-bit wide, either signed or unsigned integer type, representing the smallest addressable unit of memory on the target (1 byte). Modern processors generally agree on 8-bit bytes, and the niche use cases regarding digital signal processors fail to support modern C/C++ standards, which lead to the proposal \cite{P3477R1} to define a byte having exactly 8 bits. Therefore, we chose to adopt the same strategy for Hylo. Still, the signedness of \texttt{char}, and the exact sizes of standard integer types vary per platform.

\begin{table}[]
    \small
    \caption{\textbf{Numeric type mapping in Swift\cite{how-swift-imports-c-fundamental-types}\cite{swift-builtin-mapped-types},  Muon\cite{muon-ffi-mapping}, Zig\cite{zig-ffi-primitives}, Rust\cite{rust-ffi-primitives} and TinyGo\cite{tinycgo}.} Cells colored blue indicate that the language uses a platform-specific types, while white cells indicate a fixed type assignment that limits portability for incompatible platforms. Red cells mark unavailable or experimental mappings, and ``?''s indicate undocumented or unknown mappings.}
    \label{tab:type_mapping}
    \begin{adjustwidth}{-3cm}{} % Extends into the inner margin by 2cm

\begin{tabular}{
>{\columncolor[HTML]{D9D9D9}}l lllll}

\cellcolor[HTML]{3D85C6}{\color[HTML]{FFFFFF} \textbf{C Type Equivalent}} & \cellcolor[HTML]{3D85C6}{\color[HTML]{FFFFFF} \textit{\textbf{Swift Mapping}}} & \cellcolor[HTML]{3D85C6}{\color[HTML]{FFFFFF} \textit{\textbf{Muon Mapping}}}   & \cellcolor[HTML]{3D85C6}{\color[HTML]{FFFFFF} \textit{\textbf{Zig Mapping}}} & \cellcolor[HTML]{3D85C6}{\color[HTML]{FFFFFF} \textit{\textbf{Rust Mapping}}}                                                                                        & \cellcolor[HTML]{3D85C6}{\color[HTML]{FFFFFF} \textit{\textbf{TinyGO}}} \\
char                                                                      & CChar = Int8 (!)                                                       & \cellcolor[HTML]{CFE2F3}\textit{byte/sbyte (8-bit)}                                   & \cellcolor[HTML]{CFE2F3}\textit{c\_char = i8/u8}                           & \cellcolor[HTML]{CFE2F3}\textit{c\_char = i8/u8}                                                                                                                   & \cellcolor[HTML]{CFE2F3}\textit{int8/uint8}                           \\
signed char                                                               & CSignedChar = Int8                                                      & sbyte (8-bit)                                                                          & i8                                                                  & c\_schar = i8                                                                                                                                                                 & int8                                                           \\
unsigned char                                                             & CUnsignedChar = UInt8                                                   & byte (8-bit)                                                                           & u8                                                                  & c\_uchar = u8                                                                                                                                                                 & uint8                                                          \\
(signed) short                                                            & CShort = Int16                                                          & short (16-bit)                                                                         & \cellcolor[HTML]{CFE2F3}\textit{c\_short}                                    & c\_short = i16                                                                                                                                                       & \cellcolor[HTML]{CFE2F3}\textit{int8...int64}                         \\
unsigned short                                                            & CUnsignedShort = UInt16                                                 & ushort (16-bit)                                                                        & \cellcolor[HTML]{CFE2F3}\textit{c\_ushort}                                   & c\_ushort = u16                                                                                                                                                      & \cellcolor[HTML]{CFE2F3}\textit{uint8...uint64}                       \\ 
(signed) int                                                              & CInt = Int32                                                            & int (32-bit)                                                                           & \cellcolor[HTML]{CFE2F3}\textit{c\_int}                                      & \cellcolor[HTML]{CFE2F3}\textit{c\_int = i16/i32}                                                                                                                  & \cellcolor[HTML]{CFE2F3}\textit{int8...int64}                         \\
unsigned int                                                              & CUnsignedInt = UInt32                                                   & uint (32-bit)                                                                          & \cellcolor[HTML]{CFE2F3}\textit{c\_uint}                                     & \cellcolor[HTML]{CFE2F3}\textit{c\_uint = u16/u32}                                                                                                                 & \cellcolor[HTML]{CFE2F3}\textit{uint8...uint64}                       \\
(signed) long                                                             & \cellcolor[HTML]{CFE2F3}\textit{CLong = Int32/Int}                      & \cellcolor[HTML]{CFE2F3}\textit{int/long (32/64-bit)}                       & \cellcolor[HTML]{CFE2F3}\textit{c\_long}                                     & \cellcolor[HTML]{CFE2F3}\textit{c\_long = i32/i64}                                                                                                                 & \cellcolor[HTML]{CFE2F3}\textit{int8...int64}                         \\
unsigned long                                                             & \cellcolor[HTML]{CFE2F3}\textit{CUnsignedLong = UInt32/UInt}            & \cellcolor[HTML]{CFE2F3}\textit{uint/ulong (32/64-bit)}                       & \cellcolor[HTML]{CFE2F3}\textit{c\_ulong}                                    & \cellcolor[HTML]{CFE2F3}\textit{c\_ulong = u32/u64}                                                                                                                & \cellcolor[HTML]{CFE2F3}\textit{uint8...uint64}                       \\
(signed) long long                                                        & CLongLong = Int64                                                       &    long  (64-bit)                                                                        & \cellcolor[HTML]{CFE2F3}\textit{c\_longlong}                                 & c\_longlong = i64                                                                                                                                                    & \cellcolor[HTML]{CFE2F3}\textit{int8...int64}                         \\
unsigned long long                                                        & CUnsignedLongLong = UInt64                                              &    ulong (64-bit)                                                                               & \cellcolor[HTML]{CFE2F3}\textit{c\_ulonglong}                                & c\_ulonglong = u64                                                                                                                                                   & \cellcolor[HTML]{CFE2F3}\textit{uint8...uint64}                       \\
float                                                                     & CFloat = Float                                                          &    float (32-bit)                                                                                    & f32                                                                 & c\_float = f32                                                                                                                                                                & \cellcolor[HTML]{CFE2F3}\textit{float32/float64}                      \\
double                                                                    & CDouble = Double                                                        &    double (64-bit)                                                                       & f64                                                                 & c\_double = f64                                                                                                                                                               & \cellcolor[HTML]{CFE2F3}\textit{float32/float64}                      \\
\_Float16                                                                 & CFloat16 = Float16                                                      & \cellcolor[HTML]{F4CCCC}-                                                                                    & f16                                                                 & \cellcolor[HTML]{F4CCCC}{\href{https://doc.rust-lang.org/std/primitive.f16.html}{f16 (Exp.)}}                                                                                 & \cellcolor[HTML]{F4CCCC}-                                               \\
long double                                                               & \cellcolor[HTML]{CFE2F3}\textit{CLongDouble = Double/Float80}           & \cellcolor[HTML]{F4CCCC}-                                                                              & \cellcolor[HTML]{CFE2F3}\textit{c\_longdouble}                               & \cellcolor[HTML]{CFE2F3}{\color[HTML]{0000EE} {\href{https://docs.rs/longdouble/latest/longdouble/}{\textit{c\_longdouble}}}}                                                 & \cellcolor[HTML]{CFE2F3}\textit{float32/float64}                      \\
wchar\_t                                                                  & CWideChar = Unicode.Scalar                                              & \cellcolor[HTML]{CFE2F3}\textit{platform-defined int}                                  & ?                                                                        & \cellcolor[HTML]{CFE2F3}\textit{\begin{tabular}[c]{@{}l@{}}c\_wchar = u16/u32/\\ i32/c\_uint/c\_int\end{tabular}}                                  & \cellcolor[HTML]{F4CCCC}-                                               \\
ptrdiff\_t                                                                & Int                                                                     & \cellcolor[HTML]{CFE2F3}\textit{platform-defined int}                                  & ?                                                                        & \cellcolor[HTML]{F4CCCC}{\color[HTML]{0000EE} {\href{https://internals.rust-lang.org/t/pre-rfc-usize-is-not-size-t/15369}{c\_ptrdiff\_t (Exp.) = isize}}}                     & ?         \\
size\_t                                                                   & Int  (!)                                                                & \cellcolor[HTML]{CFE2F3}\textit{platform-defined int}                                  & usize                                                               & \cellcolor[HTML]{F4CCCC}{\color[HTML]{0000EE} {\href{https://internals.rust-lang.org/t/pre-rfc-usize-is-not-size-t/15369}{c\_size\_t (Exp.) = size}}}                                 & ?                                                                       \\
ssize\_t                                                                  & Int                                                                     & \cellcolor[HTML]{CFE2F3}\textit{platform-defined int}                                  & isize                                                               & \cellcolor[HTML]{F4CCCC}{\color[HTML]{0000EE} {\href{https://internals.rust-lang.org/t/pre-rfc-usize-is-not-size-t/15369}{c\_ssize\_t (Exp.) = isize}}}                                & ?                                                                       \\ \cline{1-6}
\end{tabular}
\end{adjustwidth}
\end{table}



A simplified mapping can be achieved when knowing the target platforms of our language. On modern desktop platforms, integer sizes are generally defined by the ILP32, LLP64 or the LP64 data model, which results in \texttt{char}: 8-bit, \texttt{short}: 16-bit, \texttt{int}: 32-bit, \texttt{long long}: 64-bit, and \texttt{long} being either 32-bit or 64-bit. Leveraging this, Swift and Muon map standard C integer types directly to their fixed-size equivalent, except \texttt{long} which is mapped to a type alias \texttt{CLong} mapping to either a 32-bit or 64-bit integer based on platform.
\footnote{
    Swift maps \texttt{char} to a signed \texttt{Int8} regardless of whether the platform defines it as signed. Zig originally mapped \texttt{char} to \texttt{u8}, but it was later decided to introduce the \texttt{c\_char} alias \cite{zig-add-cchar}.
}
\footnote{
    Swift extensively uses its signed word-sized \texttt{Int} for indices, and maps C's unsigned \texttt{size\_t} to a signed representation for convenience, relying on the assumption that programs don't use \texttt{size\_t}'s most significant bit. E.g. when importing a \texttt{size\_t} constant $2^{64} -1$ from C, Swift sees it as $-1$.
}



On the other hand, Rust, Zig -- and also Hylo -- aim to support diverse platforms, including various embedded and mobile architectures, where C integer sizes may be defined differently. Rust and Zig introduced type aliases like \texttt{c\_long} that are aliasing fixed-sized types depending on the platform. \autoref{tab:type_mapping} presents a comparison of numeric type mapping in Swift, Rust, Zig and Muon. 



In Hylo, we propose a conservative approach by default, --similar to Zig-- but with a novel addition: instead of type aliases, we use distinct types for the C standard types which cannot be implicitly converted to Hylo's fixed-width integers. If type aliases were used, C's imported integer types may accidentally match fixed integer types, in which case the interoperability would be fragile when compiling for a different platform. Instead, Hylo should require explicit conversions whenever needed that clearly express the intent of the programmer. We have identified 3 types of necessary conversions:
\begin{itemize}
    \item \textbf{trap on loss:} when the conversion is narrowing, we insert a runtime assertion that ensures that the particular input can be represented in the target type. We provide this as the default conversion, e.g. \texttt{Int32(c\_value)}. When the conversion is non-narrowing, the assertion is not needed, and a zero-extend/sign-extend operation is performed based on signedness. Additionally, this default and recommended trapping behavior could be disabled with a compiler flag to sacrifice safety for maximal runtime performance.
    \item \textbf{truncate if needed:} when the represented value's meaning allows, we can silently drop the most significant bits that don't fit the target type. Note, due to two's complement representation, values that fit in the target type will preserve their sign. This may be spelled as \texttt{Int32(truncating\_if\_needed: c\_value)}
    \item \textbf{don't narrow:} when performance is critical but we also want a compile-time guarantee that a conversion is lossless on the target platform, we can use this conversion. \texttt{Int32(non\_narrowing: c\_value)} is a conditionally enabled conversion that is only available when the conversion would be lossless.
\end{itemize}
Using these explicit conversions ensures that we explicitly capture the programmer's intent and the code remains maximally portable. The viability of the approach was validated by implementing a code generator that produces the conversion function code for the standard library, and testing its usage in Hylo code. The code generator, the generated code and an example usage can be found in TODO upload to github.


However, writing all these conversions by hand may be very inconvenient, especially when we have solid assumptions about the target platform. E.g. when writing a wrapper around the LLVM compiler, we know that it will only be used on modern desktop platforms, where we can define many of the mappings like Swift or Muon did. Therefore, we allow making project-specific explicit assumptions about the target platforms, such as \texttt{c\_short} is 16-bit. These assumptions can be checked for all the specified target platforms before compilation, so there are no risks introduced.

In addition, if a Hylo project generally uses the \texttt{Int} or \texttt{Int32} type for indices, it may specify to translate all \texttt{size\_t} function parameters to the expected types in all or some methods. Unless the bit-width is the same, this may involve truncation, which could be configured to trap on overflow or an error \emph{before} compilation. When a runtime conversion is needed, the original C function is not exposed to our Hylo code, instead a new wrapper function calling it and performing the necessary checks is exposed.

The technique of mapping a C type to its own distinct type in Hylo, and utilizing explicit conversions can not only be applied to standard integer types, but also to other types that have varying valid value ranges on different platforms: \texttt{(u)int\_fastN\_t}, \texttt{(u)int\_leastN\_t}, \texttt{intmax\_t}, \texttt{wchar\_t}, \texttt{wint\_t}, \texttt{size\_t}, \texttt{ssize\_t}\footnote{\texttt{ssize\_t} is available as a POSIX extension}, \texttt{(u)intptr\_t} and \texttt{ptrdiff\_t}.

On modern, flat-memory model systems, compilers define \texttt{size\_t}, \texttt{ssize\_t}, \texttt{(u)intptr\_t}, \texttt{ptrdiff\_t} generally as pointer-sized integers, but special architectures like capability-based architectures\cite{cheri-intro}\cite{hackernews-rust-cheri} or 16-bit segmented memory architectures\cite{arch16-bit} may have different definitions. Due to this constraint holding true on most modern platforms, we suggest mapping these types to Hylo's pointer-sized \texttt{Int}/\texttt{UInt} types by default, and providing options to override this when necessary. This generally allows for a more idiomatic use of these types in Hylo, and advanced users can still override the mapping to use distinct types if they need to.

The rest of the C integer types are more straightforward to map:
\begin{itemize}
    \item \textbf{\texttt{bool/\_Bool}} is mapped to Hylo's \texttt{Bool} type. Before C99, there was no dedicated boolean type, so conversions from \texttt{int/char} would be needed.
    \item \textbf{\texttt{(u)intN\_t} --explicit-width integers--} are mapped to Hylo's corresponding fixed-width integer types (\texttt{UInt8}, \texttt{UInt64}, etc.), which have the same memory layout and bit-width as the C types.
    \item \textbf{\texttt{\_BitInt(N)}} cannot be directly mapped to Hylo yet due to the lack of arbitrarily sized integers. However, in structs we can map them to an opaque byte sequence, exposing the value through a computed property of the closest sufficient fixed-width integer type. In function calls, a wrapper function shall be exposed that presents fixed-width integer types in parameter and return types, with the necessary conversion performed and optionally bounds-checked. 
\end{itemize} 

\subsection{Enums}
C enums are unlike enums in most other languages. In C, enums are essentially named integer constants, and the enum's type can be implicitly converted to and from the underlying integer type. However, their usage often involves specialized semantics:
\begin{enumerate}
    \item \textbf{Discriminated values:} often each enum case represents a distinct value, and a variable holding an enum value is expected to only hold one of these values. One can perform pattern matching on such values, however exhaustiveness is not checked.
    \item \textbf{Bit flags:} enums are often used to represent a set of flags, where each case is a power of two, and the enum value can be any combination of these flags. Set operations such as \textit{union}, \textit{intersection}, or membership checking may be performed via bitwise operations. 
\end{enumerate}

Assuming Hylo's enums will be similar to Swift's enum type, there are multiple possible mappings, of which the developer shall be able to choose:
\begin{itemize}
    \item \textbf{Closed enum}: a closed set of enum cases, usable for exhaustive pattern matching. When mapping libraries, we must be careful with these, as C programmers generally think of adding a new enum case as a non-breaking change. The mapping generator should assert that all enum cases are distinct.
    \item \textbf{Open enum}: an extendable set of enum cases, allowing for extensibility. This mapping is more flexible but requires a default case in pattern matching, handling any unexpected values. The generator should assert that all enum cases are distinct.
    \item \textbf{OptionSet}: a data structure with high-level operations for set operations like union, intersection, and membership checking. This mapping is suitable for bit flags, and the generator should assert that all enum cases are powers of two (not necessarily distinct).
    \item \textbf{Raw enum}: a mapping that preserves the underlying integer representation of the enum cases. This mapping is suitable for the most conservative use case, where an enum is only used to group a set of related constants.
\end{itemize}

\subsubsection{Floating Point Types}
There are 3 universally available \textbf{standard floating point types} in the C23 standard: \texttt{float}, \texttt{double}, \texttt{long double}. The standard doesn't mandate their representation, but implementations generally follow\cite{llvm-float-support} binary formats defined by IEEE 754\cite{ieee754}, except for \texttt{long double}\cite{longdouble-wiki}\cite{weird-precision-support}, thus Rust, Zig and Swift universally map \texttt{float} and \texttt{double} to IEEE 754's 32- and 64-bit binary formats correspondingly. Hylo should follow this pragmatic approach by default but also allow customization to import them as \texttt{CFloat}/\texttt{CDouble} to be more platform-agnostic, with explicit conversions where needed. These conversions may be narrowing due to rounding, overflow to $\pm\infty$ or underflow to 0, so we provide conversion functions similar to integers. Widening conversions generally preserve value

The standard also defines two \textbf{optional features}: \textbf{complex}/imaginary (since C99) and \textbf{decimal} (since C23) floating point types. While complex types may be easily implemented as a Hylo product type of the corresponding floating point types (e.g. \texttt{Complex<CLongDouble>}), on most architectures decimal types require software emulation of arithmetic operations (such as by a library like: \href{https://github.com/libdfp/libdfp}{Libdfp}), and are not yet supported in LLVM/Clang\cite{no-support-llvm-decimals}. Decimal types are also not used commonly in practice, so we also don't support them in Hylo natively.

\subsection{Composite Types}

C structs and unions are challenging to map to other languages because the new language has to replicate the memory layout of the original type, so that field access is done with correct memory offsets. Determining the struct layout involves handling the size and alignment of members, potentially including padding, and distributing bit-fields. The layout may also be influenced by compiler-specific packing attributes in all major C compilers, though their behavior is consistent.\footnote{See the effect of packing attributes in MSVC, Clang and GCC: \url{https://edu.nl/8hg98}}

C bit-fields have largely implementation-defined layout rules, which can vary significantly between compilers\footnote{See an example of varying layout of bit-fields: \url{https://godbolt.org/z/xqPEvs6YK}}. Due to the challenging nature of C bit-fields, many C interop technologies do not support interoperating with structs with bit-fields\footnote{See \href{https://github.com/rust-lang/rfcs/issues/314}{Rust's open issue about bit-field support in \texttt{\#[repr(C)]}}}\footnote{See Zig's tracking issue \href{https://github.com/ziglang/zig/issues/1499}{\#1499} for implementing bit-field support.}, and it is generally recommended to avoid using them in separately compiled library code, as interoperability is not even guaranteed between C compilers.

We identified two existing approaches for interoperating with bit-fields:

\begin{enumerate}
    \item Rust lets us \textbf{annotate Rust structs, unions and enums --that have C-compatible members--} with the \texttt{\#[repr(C)]} attribute. Rust then lays out these members according to the C ABI by reimplemented logic\cite{rust-c-layouting}. Notably, bit-fields are not supported natively.
    
    However, Bindgen\cite{rust-bindgen}, an external binding generator for Rust, attempts to emit Rust \texttt{\#[repr(C)]} structs from C header files, complementing the layout algorithm with a custom bit-field handling algorithm, and exposing accessor methods that let the user operate with a correctly aligned and padded Rust integer type. This method can often work but has limitations and bugs, a notable problem being that C bit-fields may share storage with other members' padding bits, which is challenging to replicate when generating separate struct fields for each member.\footnote{See \href{https://github.com/rust-lang/rust-bindgen/issues/743\#issuecomment-321051199}{issue \#743 in Bindgen's repository} related to the overlapping storage bug.}.
    
    \item Swift imports C declarations by \textbf{directly embedding Clang into the compiler}, and transforming declarations from Clang AST directly to Swift AST declarations\footnote{See structs being imported to Swift in \href{https://github.com/swiftlang/swift/blob/9a0a831b0198e1b794a66316487aacef3d692ca4/lib/ClangImporter/ImportDecl.cpp\#L2079}{ImportDecl.cpp at line 2079}}. It synthesizes computed properties for accessing C bit-fields, which present an API as if the bit-fields were regular Swift fields while performing the necessary bit-masking and shifting to access the individual fields\cite{how-swift-imports-c-structs}. Swift preserves the references to the original Clang AST nodes, which can be used when compiling member access to find out the exact member offsets, therefore resulting in a robust mapping.
\end{enumerate}

The first approach, decoupling the mapping of implementation-defined C constructs from the main compiler is useful for reducing the compiler's binary size (no need to embed Clang in the general case), simplifies the development environment setup for both the compiler authors and its users, and keeps the bugs easier to fix in the substantially smaller codebase of the binding generator. On the other hand, embedding Clang and directly maintaining layout information from the Clang AST for the backend reduces the maintenance burden of the binding generator, and allows for robust handling of implementation-defined C constructs.

We propose a hybrid approach for Hylo that combines advantages from both methods. Instead of letting the Hylo compiler take care of C struct layout, we map members of unions and structs to a Hylo struct containing a \textbf{contiguous inline byte sequence} with the same length as the original C type's size (including padding bytes). Subsequently, we synthesize computed properties (subscripts) for accessing the individual members, which provide idiomatic access to the members, exactly as if they were stored properties. For struct members, we can leverage LibClang's functions (e.g. \texttt{clang\_Cursor\_getOffsetOfField}) to get the offset, the size in bytes of the underlying type and the bit-width in case of bit-fields. Union members always start a  t the same memory address, without an offset\cite{c23-struct-and-union-specifiers}.

To ensure the same data layout in Hylo as in C, we need an additional annotations for the Hylo struct: we need to specify its \textit{alignment}, so that the Hylo compiler can know on which boundary it should allocate instances of the struct.

\paragraph{
Flexible Array Member
    }
    \label{par:fam}
    A special array member without a specified size may appear at the end of a struct declaration. Such member can get more storage based on how much space is allocated for the struct, which allows accurate representations of e.g. network packets composed of a header and a variable length payload. This may be implemented in Hylo as a computed property that exposes a typed pointer to the first element of the flexible array member. TODO finish FAM, why are they not UB. Structs with a flexible array member shall additionally have a non-copyable zero-sized member that prevents the automatic synthesis of the \texttt{Copyable} and \texttt{Movable} traits.

\subsection{Pointer Types}
C's \textbf{pointer to data} are mapped to Hylo's \texttt{Pointer<T>} and \texttt{PointerToMutable<T>} types, depending on the \texttt{const}ness of the pointee. \textbf{Function pointers} can be mapped to Hylo closure types with a \texttt{@convention(C)} attribute and an empty environment, which ensures that the correct calling convention is used when the function is called. The details of this are discussed in the proposal \cite{hylo-function-pointers}.

For indicating nullability of pointers, Hylo could adopt a similar approach to Swift\cite{how-swift-imports-c-nullable}, where C declarations are annotated with a \textit{nullable}, \textit{non-nullable} or \textit{null-unspecified} attribute. A \textit{nullable} annotation would wrap the imported Hylo pointer/closure within an optional type, which must be guaranteed the same memory layout as the original C pointer type, and where unwrapping the optional involves checking for null under the hood.

Swift already supports a basic scoped customizability using \texttt{pragma}s but the mapping should be more flexible and be possible to set from outside of the header files.

\subsection{Type Qualifiers}
C types may have 0 or more type qualifiers that modify their semantics or the way their usage gets compiled. \texttt{const}-qualified types indicate that their value cannot be modified, at least not through that particular binding. This is widely applicable to language constructs that need to be mapped, such as global variable bindings, function parameters, struct members, and nested types within pointers. Constant global variables and struct members are mapped to immutable let bindings, function parameters (as always) are mapped to immutable parameters, and pointers to const types are mapped to \texttt{Pointer<T>} instead of \texttt{PointerToMutable<T>}.

The \texttt{volatile}, \texttt{restrict} and \texttt{\_Atomic} qualifiers signal additional information for the compiler backend. Further research is necessary to see how to handle these properly, as many interop technologies ignore them (Swift, Rust Bindgen) while preliminary efforts for support exist in Zig\cite{zig-qualifiers}.

\subsection{Global Variables and Constants}
While Hylo deliberately avoids shared mutable state due to its inherent risks \cite{sharedmut}\cite{shared-mutable-state}, practical needs in C interoperability and embedded systems, such as sharing data with interrupt service routines \cite{rust-embedded-pain}, require its introduction. We propose declaring mutable globals with \texttt{unsafe var <name>: <type>}, mandating that all access occurs within an \texttt{unsafe} context to ensure developer vigilance. The \texttt{@extern} version, \texttt{@extern unsafe var <name>: <CType>}, supports linking with external C code. In contrast, their immutable counterparts are declared using \texttt{let} and \texttt{@extern let} for global constants, and can always be accessed safely. Support for thread-local variables is not yet proposed and remains an area for future research.

\subsection{Function Declarations}
C functions are mapped to \texttt{@extern} Hylo function declarations with an additional \texttt{@convention(C)} attribute so that the compiler can use the correct calling convention when calling the function.

C function signatures are often ambiguous; a pointer parameter might be used for reading, writing, or both. This ambiguity presents a challenge because while Hylo's own parameter conventions (\texttt{sink}, \texttt{let}, \texttt{inout}, \texttt{set}) could precisely capture these different intents, automatically and safely inferring the correct one from a C header is infeasible, and often impossible.

Therefore, Hylo adopts a consistent default inspired by Rust and Swift. All C functions are imported with their parameters treated as immutable \texttt{let} inputs. A C pointer is handled as a value--the memory address—which is passed immutably and translated to a \texttt{Pointer<T>} or \texttt{PointerToMutable<T>}. Consequently, the imported function cannot modify its arguments' values. It can only modify the data \textit{at} the location a \texttt{PointerToMutable<T>} refers to. Notably, the calling convention 

It is then the developer's responsibility to build safe, idiomatic Hylo wrappers around these raw, low-level imports to restore clear value semantics.

\subsubsection{Header-Defined Functions (\texttt{static}, \texttt{inline}, \texttt{static inline})}
\label{handling_inline_c_functions}

Functions defined entirely within header files, using the \texttt{static}, \texttt{inline} or \texttt{static inline} keywords, pose a significant challenge because they are not guaranteed to produce linkable symbols in object files. To call them from another language, their function bodies must be made available. Our analysis of existing technologies reveals two primary architectural approaches to this problem:

A significant challenge in C interop is handling functions defined entirely within header files. These functions, declared with the \texttt{static}, \texttt{inline}, or \texttt{static inline} specifiers, lack a stable, externally visible symbol in the object file, making them non-linkable by default. To invoke them from another language, their implementation must be made accessible. Our analysis reveals two primary architectural approaches to this problem:

\textbf{Generating C Wrappers.} This strategy involves creating a new \texttt{.c} source file that contains simple, externally visible wrapper functions for each desired header-defined function. This file is then compiled alongside the original library, and the foreign language binds to the newly created symbols with unique names (e.g., suffixed with \texttt{\_\_wrapper}). Performance recovery via Link-Time Optimization (LTO) is possible but remains unreliable. It requires a shared compiler backend like LLVM, and even then, the inlining is a heuristic optimization that may not be performed.

This technique is employed by Rust's Bindgen, though it currently has (unnecessary) limitations regarding plain \texttt{inline} functions\cite{bindgen-inline-limitation} that we proposed to mitigate in\cite{bindgen-inline-proposal}.

\textbf{Direct LLVM IR Emission:} A more deeply integrated approach, employed by Swift-C interop, is to use Clang as a library to request the compilation of the header-defined function directly into LLVM Intermediate Representation. LLVM IR can then be consumed by the new language's backend, given that it uses LLVM. By operating at the compiler level rather than the linker level, this strategy eliminates the cost of function calls.

For Hylo, a language early in its development, we adopt the wrapper-generation approach. This prioritizes compiler simplicity and avoids a hard dependency on a specific C compiler front-end or a single backend like LLVM, which we consider a prudent trade-off at this stage.


\subsubsection{Variadic Functions}
Variadic functions in C can be called with any number and type of arguments without their usage being checked at compile time or runtime. Due to their inherent lack of type safety, they shouldn't be directly imported into Hylo. Instead of adding a special language feature exclusively for C interoperability, as Rust did for its FFI\cite{rust-variadic-ffi}, Hylo prioritizes keeping the language simple.

Since variadic functions are common in C code, Hylo supports two primary strategies for interoperability. The first approach is to write a dedicated, type-safe wrapper in C. Alternatively, developers can leverage the common C practice of calling a function variant that accepts a \texttt{va\_list} pointer\cite{wrapping-valist}. Hylo may facilitate this by providing a mechanism to explicitly construct a \texttt{va\_list} object\cite{swift-variadic-ffi}. Although this method remains fundamentally type-unsafe, its manual nature makes the associated risks explicit. This design choice enables necessary interoperability without creating a misleading abstraction that would disguise an unsafe C pattern as a conventional, safe Hylo function call.

\subsection{Synthesizing Core Trait Conformances}

Hylo offers a set of core traits, including \texttt{Copyable}, \texttt{Movable} and \texttt{Deinitializable}, that provide ways to specify and customize a type's lifetime management capabilities. To add these capabilities to a standard Hylo struct, a developer must explicitly declare conformance to the desired traits. The compiler can then automatically synthesize the necessary implementations, provided that all of the struct's members also conform to those traits, or it can be customized by providing an implementation.

While it might seem practical to apply this synthesis to C types—for instance, by using a heuristic such as checking if a C struct contains any pointers—doing so would introduce hidden, implementation-defined promises that could hinder library evolution. A developer must remain conscious of the capabilities assigned to imported C types and determine whether a default implementation is sufficient or if a custom one is required.

This challenge is evident in existing tools. For example, Rust's bindgen attempts to derive \texttt{Copy} and \texttt{Clone} traits automatically on a best-effort basis, allowing users to opt out as needed\cite{bindgen-nocopy}. However, this approach can lead to incorrect value semantics\footnote{This may seem like an insignificant issue in practice, as most C functions take struct parameters by pointers, but within a Hylo context, we expect that the majority of wrapping code can deal with values, and only expose their pointers underlying memory address for duration of the C function call.}, such as when a type containing pointers to heap-allocated data is mistakenly marked as copyable\footnote{See an example for an incorrectly derived Copy/Clone conformance: \url{https://edu.nl/ghfrm}}.

Therefore, we propose a more deliberate approach for Hylo. Developers can either write the trait conformances for C types manually or use an auxiliary tool to generate the declarations based on conservative heuristics. Regardless of the method chosen, we encourage developers to check the resulting code into their repository and take ownership of it. This practice avoids integrating trait derivations into the build process directly. Should subsequent changes to the original C declarations—such as adding a pointer to a struct—invalidate the initial assumptions, the compiler can emit a warning to alert the developer as proposed in \cite{hylo-trait-ptr-conformance-warning}.

\subsection{Macros}
C preprocessor macros present a major challenge for high-fidelity interoperability due to their purely textual nature.  Unlike functions, macros operate on token streams before the C compiler performs semantic analysis, leading to well-known issues with operator precedence, unintended duplication of side effects, and a complete lack of type information.  While some simple use cases can be handled, the direct translation of more complex use cases is an open research problem. Consequently, devising a macro translation strategy for Hylo is considered out of scope for this paper, however we provide a brief overview of existing industry and research efforts in \autoref{ssec:macro}.

\subsection{Array Types}
In C, arrays are contiguous sequences of elements that can appear in several contexts\cite{carrays}, each requiring a specific mapping strategy in Hylo.

\textbf{Arrays of Constant Known Size:} These arrays can be defined as global variables, members of structs, or as part of multi-dimensional array types. This C type directly corresponds to Hylo's \texttt{Buffer} type, which represents a fixed-size array with its elements stored inline.

\textbf{Arrays of Unknown Size:} These can be found in a few specific situations: as incomplete types at file scope (a case not currently supported), as flexible array members in structs (see \autoref{par:fam}), and in function parameters. When used as function arguments, C arrays--whether of known or unknown size--decay into a pointer to their first element. While this process loses compile-time information about the array's length, it preserves the element's type information, which is crucial for calculating the correct offset during pointer indexing. Consequently, arrays in function parameters are mapped to Hylo's \texttt{Pointer<T>} or \texttt{PointerToMutable<T>}, depending on the \texttt{const}-qualification of the element type.

\textbf{Variable Length Arrays:}
C supports a highly constrained form of dependent type called variable length arrays (VLAs) that allows the size of an array type to be determined at runtime. This is especially useful for multidimensional data\cite{vla-reddit} where the length of each dimension can vary but we still want to allocate the data contiguously and use indexing to access the elements: \\
\texttt{int \textbf{arr}[10][n][m] = \textbf{malloc}(10 * n * m * \textbf{sizeof}(int)); \space\space\space   \textbf{arr}[1][2][3] = 42;}

Concerning imported declarations, VLAs may only appear in function parameters, where their corresponding dimensions are also specified, allowing the implementation to use those to calculate the accurate offsets for indexing. Hylo doesn't have a concept of VLAs in the type system, but for interoperability purposes it's enough to expose VLA parameters as \texttt{Pointer<T>}/\texttt{PointerToMutable<T>} parameters, where \texttt{T} is the underlying type of the smallest unit element in the VLA (\texttt{int} in the example above). This lets the user allocate and set up a VLA parameter in Hylo using manual offset calculations for indexing, and pass the pointer to the C function.
