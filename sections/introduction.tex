\newpage
\section{Introduction}

The Hylo programming language aims to provide the performance of C++ with a simpler, safe programming model based on mutable value semantics and a powerful generics system. For a new systems language, the ability to leverage the vast ecosystem of existing libraries and operating system APIs can significantly accelerate its adoption by reducing the barrier to entry. This allows new users to solve real-world problems from the start, rather than first needing to implement foundational tools like a GUI toolkit or network protocol. Due to its relatively simple and stable Application Binary Interface (ABI)—the low-level conventions for how functions are called and data is represented in machine code—C has become the de facto bridge language between languages. Consequently, a high-fidelity C interoperability solution represents a critical component for Hylo's practical success.

However, designing such a layer is fraught with complexity. It demands a deep understanding of both the source and target languages, platform-specific ABI conventions, and the nuances of various C dialects and compiler extensions which introduce non-standard features or memory layouts that an interoperability tool must understand to avoid undefined behavior. Existing solutions differ widely in their capabilities, often forcing a trade-off between portability and ease-of-use, and frequently impose a rigid or hard-to-customize mapping for C constructs that may have multiple valid interpretations in the new language. A lack of a comprehensive analysis of the design space creates a barrier for new languages, often leading to ad-hoc solutions that suffer from bugs or limitations.

In this paper, drawing from a methodology that combines a review of industry technologies, academic literature, expert interviews, and targeted prototype implementations, we present a principled technical design for high-fidelity C interoperability for Hylo, and answer the main research question: \textbf{How can a modern systems language with an emphasis on safety, simplicity and performance approach seamless C interoperability?}

Our contributions are:

\begin{enumerate}
    \item \textbf{A set of desirable properties for a high-fidelity C interoperability solution}, derived from an analysis of primary use cases and a review of existing state-of-the-art technologies. (\Autoref{sec:desirable_properties})   
    \item \textbf{A detailed architectural design for Hylo's C interoperability tooling.} This includes a comparative analysis of C header parsing technologies, a proposed architecture for compiling C function calls by leveraging Clang and LLVM, and a plan for providing transparent cross-language IDE features. (\Autoref{sec:architectural_design})
    \item \textbf{A comprehensive specification for mapping C constructs to Hylo.} This defines the semantic translation of primitive and composite types, pointers, and complex constructs like macros and inline functions. Crucially, it includes a design for mapping C's "semantic dialects" (e.g., enums used as bit flags) to stronger, idiomatic Hylo types, validated in part by targeted prototypes. (\Autoref{sec:mapping_c_constructs})
\end{enumerate}

The remainder of this paper is structured as follows. Section 2 reviews the state-of-the-art and defines the desirable properties of a C interoperability layer. Section 3 presents our architectural design, followed by Section 4, which details the complete C-to-Hylo mapping specification. Section 5 validates our design through targeted prototypes. Finally, Section 6 discusses the implications and future work before we conclude."

\textit{TODO add citations and update references in summary.} 
% \begin{itemize}
    % \item Problem: C is lingua franca between languages, so one would expect it is suitable for it. No, it's horrible. Large historical baggage of ABI differences, type sizes, implementing interop solutions is always costly or not very portable. 
    
    
    % $\rightarrow$ we analyze what are the options for handling this and what are the expected costs of different approaches.
    
    % \item We analyzed the use cases of C/C++ interoperability (add condensed version of interop motivations), design a desirable roadmap for Hylo, a systems programming language (here describe what Hylo is, and what are its characteristics).
    
    % \item C interoperability tools are often very limited in capabilities, which makes them a usability nightmare, leading to inconsistencies bugs, and security issues.
    % $\rightarrow$ Swift achieved seamless cross-language interop, analyze the magic behind it
    % \item Little consideration of the practical and technical aspects in academic research
    

    % \item State the research questions again, clearly, briefly

% \end{itemize}






% What are we writing about?
% - A principled technical design of C interoperability and its required tools for Hylo
% - mapping C constructs to Hylo, a safe systems programming language.

% Why are we writing about it?
% - Existing solutions are differing in terms of capabilities, usability and customizability
% - Designing cross-language interop is complex: one must understand the source and the target language well, know about platform differences, ABI guarantees, and have a good understanding of the use cases and requirements for the interop layer.
% - No comprehensive analysis of design considerations, it increases the barrier to entry for new languages, or leads to un-portable interoperability solutions in the interest of time.
% - Usability issues and unsound interoperability solutions lead to inconsistencies, bugs, and security issues.

% Why now?
% - As Hylo is being developed, it needs to leverage existing libraries to bootstrap its ecosystem and adoption.
% - Interoperating with a language requires the knowledge about its calling conventions and ABI on the target platform, and C is still the only statically compiled language that has a stable ABI specified across all platforms. All major languages have a C interop solution because of this, which made C the lingua franca (bridge language) of programming languages. (other forms of exist, such as virtual machine based solutions, and some platform-specific solutions for Swift and C++ but they are not as widely adopted - idk if we need to mention this in the intro)



% RQ1: What are the desired properties of a high-fidelity C interoperability layer for a modern systems language?
%     Contribution 1: Use cases and corresponding requirements for C interoperability.

% RQ2: How can C language constructs—including types, functions, macros and common programming idioms—be mapped to Hylo constructs (if even desirable), and what tooling architecture is required to automate this translation?
%     Contribution 2: A detailed architectural design for Hylo's C interoperability tooling. This includes a comparative analysis and justified selection of a C header parsing technology (e.g., LibClang), a proposed architecture for lowering function calls, and a plan for providing transparent cross-language IDE features.

%     Contribution 3: A comprehensive specification for mapping C declarations to Hylo. It details the mapping of primitive types (validated by the C integer conversion prototype), structural types, pointer types and macros. Crucially, it includes a design for mapping C's "semantic dialects," such as enums used as bitflags, to stronger, more idiomatic Hylo types.

\bigbreak \bigbreak \bigbreak
