\section {Background: Use Cases of C/C++ Interoperability}
\label{sec:use_cases_of_interop}
TODO: write introduction, add references to Rust, Swift, Zig and D for their suitability in different use cases within the sections.

\subsection{Leveraging OS APIs}

Interfacing with the operating system is essential for hosted programs to manage hardware, system resources, and processes. Major operating systems like Linux, Windows, and macOS offer stable C APIs that enable tight integration. Due to the relative simplicity of the C ABI, C provides the most straightforward means of OS interaction for statically compiled languages.

For many common operations, such as file I/O and memory management, the LibC, \textbf{C standard library} \cite{libc} offers a portable abstraction layer over OS-specific functionalities, facilitating the development of platform-independent code. While the future Hylo standard library may involve native OS representations, leveraging the ubiquity of LibC allows Hylo to be used in a wide range of environments without requiring extensive platform-specific implementations.

\subsection{Bootstrapping a Library Ecosystem}
Bootstrapping a library ecosystem is hard. While most programs would be expressible in idiomatic Hylo code, new users typically want to use Hylo to solve their own problems instead of implementing a network protocol or a GUI toolkit. The ability to use C libraries from Hylo greatly \textbf{reduces the barrier to entry}, as the first set of Hylo libraries can be lightweight, safe wrappers around existing libraries that expose a C API. This way, Hylo can be used to solve real-world problems right from the start, and the ecosystem can grow organically as more libraries are written in Hylo.

Direct interoperability with other higher-level systems programming languages like Rust, C++ or Swift could further ease the wrapping process, as these languages provide safer abstractions and more static guarantees than C.

\subsection{Hylo as a C++ Successor}
Hylo's goal is to prove that mutable value semantics can provide safety and simplicity without compromising the performance of C and C++\cite{hylo-lang}. This requires that most problems that C++ is used for can be expressed in Hylo efficiently. While our hypothesis is that Hylo will be great for starting new projects, integrating into an existing project without rearchitecting it to avoid aliasing mutable state is challenging  due to Hylo's stricter semantics.

An ideal successor language would allow \textbf{unordered, incremental adoption} with bi-directional interoperability\cite{requirements-for-cpp-successor-languages}, meaning that C++ components could be gradually replaced with Hylo components without having to migrate all its dependencies first. Carbon and Swift aim to support this with features like allowing Carbon types to inherit from C++ types\cite{carbon-interop-goals} or allowing to conform C++ types to Swift protocols\cite{swift-cpp-posthoc-conformance}, however there are many limitations, and this is still an open research area. TODO add citation to Swift limitations, and mention Zngur's approach.

There is also an apparent \textbf{trade-off between the expressiveness of the mapped subset of C++ and the "purity" of the new language}. Mapping foreign constructs may require the new language to be extended with language or library features that circumvent the safety guarantees and value semantics. If the usage of such constructs is only allowed in an unsafe context, much of the Hylo code interacting with the C++ side would need to be marked as unsafe, which would increase the noise without providing much value. (Todo relate this to Zngur's approach. Maybe no need to talk about solutions here though, but Zngur's approach is quite a big mindset shift.)

\textbf{Integration to the existing developer ecosystem}, from IDEs to build is also crucial for commercial adoption in large projects. Debugging, profiling, testing workflows and language server features across boundaries shall be supported.\cite{requirements-for-cpp-successor-languages} If Hylo will have its own build systems, it shall shall support building existing C++ dependencies, but building Hylo code shall be supported in existing C++ build systems like CMake, Bazel and Meson to ease adoption in existing projects.

\subsection{Authoring C++ Libraries In Hylo}
Hylo offers a more ergonomic and less error-prone approach to generic programming than C++ through its robust, declaration-site-checked generics system. Many of Hylo's features, with a few exceptions like subscripts, could translate to idiomatic C++ code. This opens the door to emitting highly usable C++ libraries from Hylo.

These Hylo-generated libraries would adhere to C++ best practices, including leveraging concepts (derived from Hylo's traits) to ensure misuse is difficult. Using Hylo as a frontend for library development could mitigate common C++ authoring pitfalls, such as the inability to precisely specify function APIs and the reliance on creative template instantiation to expose bugs.

Some limitations arise due to the differences between the generics systems' semantics, e.g. with regards to specializations, whose implications would need further exploration.

TODO: mention similar technologies that do this.

\subsection{Exposing Hylo Libraries Through a C API}

Once a valuable library is written in Hylo and becomes popular, other language ecosystems may want to use it, e.g. for efficiency reasons or because rewriting it in the other language and maintaining the port would be infeasible. To support interfacing with all other languages without dedicated direct interoperability mechanism, it's best to expose Hylo libraries through a C API - which \href{https://arxiv.org/abs/2411.08388}{most languages support interfacing with}, at least to a practical extent. Other languages can then write wrappers around the C API to provide a more idiomatic interface.


Exposing a C API may be done in various ways:
\begin{enumerate}[label=(\alph*)]
\item \textbf{Manually}
  \begin{itemize}
  \item You need to make sure that your data structures and functions are marked with C ABI annotations. This may not align with the internals of your library, and you might not want to have C representation in the general case when the library is consumed from Hylo, so some manual wrapping may be necessary. You also need to write the matching C/C++ headers manually.
  \item See Rust's \texttt{reprc(c)} for data structures and \texttt{extern "C"} function declarations, and Swift's experimental \texttt{@c\_decl}.
  \end{itemize}
\item \textbf{Semi-automatically}
  \begin{itemize}
  \item You still need to annotate your exposed data structures and functions as C compatible, but the header files are automatically generated based on your public C-compatible declarations. See Rust's CBindgen.
  \item Example: C - Go interop (CGO)
  \item This is a desirable improvement over the manual approach, but it can be added on top of the manual approach without requiring any changes to the language (see Rust's CBindgen). %todo add ref
  \end{itemize}
\item \textbf{Automatically}
  \begin{itemize}
  \item When writing your library, you don't need to think about making your structs and functions C-compatible - everything is handled for you automatically. This is hard for more complex languages, see concerns raised on this forum: \url{https://forums.swift.org/t/formalizing-cdecl/40677}. (todo)
  \item Since this C API will likely have many users, it is crucial to not make this a fragile API. The developer should be notified about C ABI breakages when publishing a new release.
  \item Examples:
    \begin{itemize}
    \item Fusion (transpilation to C)
    \item Embedded C scripting languages (TODO)
    \end{itemize}
  \end{itemize}
\end{enumerate}


Some technologies also decided to not support declaring C structs in the new language, just import them from a C header file. Despite the loss of flexibility in where we write our APIs, this way, developers can still have fine control over what is being exposed and it also spares implementation effort for the interop technology maintainers, as importing C headers is likely already implemented due to its higher priority.


\subsection{Authoring Hylo libraries that can be directly used from other languages}
\begin{itemize}
\item It's the safest and least error-prone to expose a Hylo library without having to rely on C as an intermediary. UniFFI-rs (Rust) provides an example of this.
\end{itemize}
\subsection{Use Cases Discarded}
\begin{itemize}
  \item \textbf{Hylo as an embedded scripting language:} multiple embedded scripting languages support great interoperability with C++ such as Hobbes and ChaiScript. %todo reference
  However, Hylo is designed to be beneficial for medium to large scale projects, and would provide no significant benefit over existing embedded scripting languages, so we can discard this use case.
  \item \textbf{C/C++ Interop for additional performance:} Hylo is designed as a high-performance system programming language, meaning that unlike managed languages (e.g. Python, NodeJS), it doesn't need to interoperate with C/C++ for improved performance.
  \item \textbf{Hylo as a C Successor:} While Hylo aims to leave no room for a lower-level language to achieve additional performance, its design discourages unsafe operations through reduced ergonomics, such as the absence of a dedicated pointer syntax. For problems inherently requiring frequent unsafe operations, this design choice could hinder productivity without offering significant advantages over simpler languages like Zig, C3, or Odin. In such scenarios, these alternatives might prove more suitable, and C interoperability could be used to expose their functionalities to Hylo.

\item \textbf{Evolving C++}\\
  C++'s unopinionated design—its greatest strength—lets developers express nearly anything, supporting multiple programming paradigms. This ensures its continued relevance. However, decades of rapid complexity growth have made the language increasingly difficult to manage.

  %$ todo add reference to circle and carbon
  While efforts like Carbon and Circle aim to evolve C++, Hylo takes a different approach. Instead of directly extending C++, Hylo offers a more constrained programming model, prioritizing safety and simplicity. We believe Hylo can solve the majority of C++'s use cases with greater simplicity, safety, and consistency. Like Rust, certain problems will remain easier to tackle in C++, but our goal is to minimize that subset.
\end{itemize}


