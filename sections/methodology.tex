\section{Methodology}
\label{sec:methodology}
This research was conducted using a combination of academic and industry research, expert interviews, implementing prototypes, and feedback from the Hylo developer community. The goal was to gain a deep understanding on the state of the art in C interoperability technologies, and to develop targeted prototypes that demonstrate the feasibility of solving key challenges. 

\subsection{Industry Research}
We started by reviewing the existing C/C++\footnote{C++ was included for perspective on Hylo's long-term goal of C++ interoperability.} interoperability technologies in various programming languages, including Swift, Rust, D, Carbon, and Zenon. This involved reading documentation, blog posts, and watching conference talks to understand their features, limitations, and collect desirable properties for an interoperability solution for Hylo.

We also conducted interviews with language experts from the C++, Rust, Zenon and Hylo language communities. These interviews were conducted in an open format, allowing us to explore specific topics in depth while also gathering general insights on the challenges and solutions related to C (and C++) interoperability. We conducted preliminary discussions within the Hylo community, and gathered feedback on proposed solutions.

\subsection{Academic Literature Review}
We conducted an exploratory academic literature review using a combination of Scopus queries, snowballing from (todo ref), and AI-assisted tools like Elicit and Undermind. The exact prompts are presented in TODO appendix.

The Scopus based search was done in 3 parts: general papers related to cross-language interoperability, papers related to C interoperability, and papers related to C++ interoperability. The search was limited to the last 10 years, as the industry review suggested that most of the significant advancements in high-fidelity interoperability have been made since the introduction of Swift and Rust since 2014 and 2015 respectively. However, snowballing revealed some relevant papers from earlier years even from 1996. Furthermore, an additional Undermind-based search was conducted regarding C macro translation and interoperability, whose results we present in \autoref{sec:related_work}

\subsection{Prototyping}
To validate design decisions and explore technical feasibility, we implemented three proof-of-concept prototypes. Each addressed a specific challenge in C interoperability: the conversion of integer types, the mapping of complex C data structures, and the introspection of the C Application Binary Interface (ABI).

The first prototype for \textbf{C integer conversions} implemented the design detailed in \autoref{ssec:integer_types_mapping}. With the aim of developing a solution that could be upstreamed to the Hylo compiler, this work involved API design, code generation for the $N^2$ conversions, extending compiler intrinsics, and implementing the LLVM backend lowering.

We then developed prototypes to validate a strategy for representing C data structures without coupling the Hylo compiler to a C compiler. This involved two interconnected efforts. First, we built the \textbf{ABI Explorer}, a standalone Swift tool exposed in a web interface\cite{abi-explorer} that uses LibClang's C API to inspect C struct layouts (including field offsets, sizes, and alignment) for any target platform. This tool provided the necessary data to implement a series of \textbf{mapping prototypes} for challenging C constructs, namely unions\cite{hylo-union-mapping}, flexible array members\cite{hylo-fam-mapping}, and bit-fields\cite{hylo-bit-field-mapping}. The correctness of these mappings was verified using unit and property-based tests.